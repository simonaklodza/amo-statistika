%% LyX 2.1.3 created this file.  For more info, see http://www.lyx.org/.
%% Do not edit unless you really know what you are doing.

\documentclass{article}\usepackage[]{graphicx}\usepackage[]{color}
%% maxwidth is the original width if it is less than linewidth
%% otherwise use linewidth (to make sure the graphics do not exceed the margin)
\makeatletter
\def\maxwidth{ %
  \ifdim\Gin@nat@width>\linewidth
    \linewidth
  \else
    \Gin@nat@width
  \fi
}
\makeatother

\definecolor{fgcolor}{rgb}{0.345, 0.345, 0.345}
\newcommand{\hlnum}[1]{\textcolor[rgb]{0.686,0.059,0.569}{#1}}%
\newcommand{\hlstr}[1]{\textcolor[rgb]{0.192,0.494,0.8}{#1}}%
\newcommand{\hlcom}[1]{\textcolor[rgb]{0.678,0.584,0.686}{\textit{#1}}}%
\newcommand{\hlopt}[1]{\textcolor[rgb]{0,0,0}{#1}}%
\newcommand{\hlstd}[1]{\textcolor[rgb]{0.345,0.345,0.345}{#1}}%
\newcommand{\hlkwa}[1]{\textcolor[rgb]{0.161,0.373,0.58}{\textbf{#1}}}%
\newcommand{\hlkwb}[1]{\textcolor[rgb]{0.69,0.353,0.396}{#1}}%
\newcommand{\hlkwc}[1]{\textcolor[rgb]{0.333,0.667,0.333}{#1}}%
\newcommand{\hlkwd}[1]{\textcolor[rgb]{0.737,0.353,0.396}{\textbf{#1}}}%

\usepackage{framed}
\makeatletter
\newenvironment{kframe}{%
 \def\at@end@of@kframe{}%
 \ifinner\ifhmode%
  \def\at@end@of@kframe{\end{minipage}}%
  \begin{minipage}{\columnwidth}%
 \fi\fi%
 \def\FrameCommand##1{\hskip\@totalleftmargin \hskip-\fboxsep
 \colorbox{shadecolor}{##1}\hskip-\fboxsep
     % There is no \\@totalrightmargin, so:
     \hskip-\linewidth \hskip-\@totalleftmargin \hskip\columnwidth}%
 \MakeFramed {\advance\hsize-\width
   \@totalleftmargin\z@ \linewidth\hsize
   \@setminipage}}%
 {\par\unskip\endMakeFramed%
 \at@end@of@kframe}
\makeatother

\definecolor{shadecolor}{rgb}{.97, .97, .97}
\definecolor{messagecolor}{rgb}{0, 0, 0}
\definecolor{warningcolor}{rgb}{1, 0, 1}
\definecolor{errorcolor}{rgb}{1, 0, 0}
\newenvironment{knitrout}{}{} % an empty environment to be redefined in TeX

\usepackage{alltt} 
\usepackage{ucs}
\usepackage[utf8x]{inputenc}
\usepackage[sc]{mathpazo}
\usepackage[T1]{fontenc}
\usepackage{geometry}

\newenvironment{uzdevums}[1][\unskip]{%
\vspace{3mm}
\noindent
\textbf{#1 uzdevums:}
\noindent}
{}

\geometry{verbose,tmargin=2.5cm,bmargin=2.5cm,lmargin=2.5cm,rmargin=2.5cm}
\setcounter{secnumdepth}{2}
\setcounter{tocdepth}{2}
\usepackage{url}
\usepackage[unicode=true,pdfusetitle,
 bookmarks=true,bookmarksnumbered=true,bookmarksopen=true,bookmarksopenlevel=2,
 breaklinks=false,pdfborder={0 0 1},backref=false,colorlinks=false]
 {hyperref}
\renewcommand{\abstractname}{Anotācija}
\hypersetup{
 pdfstartview={XYZ null null 1}}
\IfFileExists{upquote.sty}{\usepackage{upquote}}{}
\begin{document}





\title{43.\ AMO rezultāti: Valmieras Valsts ǧimnāzija}

\author{LU Neklātienes matemātikas skola, \texttt{nms@lu.lv}}
\date{2016-04-13}

\maketitle

\section{Aktivitātes līmenis}

Aktivitātes procents katrā no skolām. (AMO dalībnieku skaita
attiecība pret visu skolēnu skaitu attiecīgajā skolā).

\begin{itemize}
\item Atbilstoši skolas urbanizācijas tipam 
izveidota rangu tabula ar aktivitāšu procentiem dilstošā secībā. 
\item Skolas statistiskajam re\v{g}ionam uzzīmēta joslu diagramma 
({\em bar chart}), kurā skolas aktivitātes stabiņš izcelts ar citu krāsu. 
\item Aktivitāšu tabula 119 novadiem/pilsētām
\item Aktivitātes karte 119 novadiem/pilsētām - krāsojums intensīvāks tur, kur aktivitāte lielāka.
\end{itemize}


\begin{knitrout}
\definecolor{shadecolor}{rgb}{0.969, 0.969, 0.969}\color{fgcolor}

{\centering \includegraphics[width=\maxwidth]{figure/minimal-regional-activity-1} 

}



\end{knitrout}





\begin{knitrout}
\definecolor{shadecolor}{rgb}{0.969, 0.969, 0.969}\color{fgcolor}

{\centering \includegraphics[width=\maxwidth]{figure/minimal-aplitis-1} 

}



\end{knitrout}


\section{Skolas savāktie punkti}

\begin{enumerate}
\item Skolas urbanizācijas tipam -- rangu tabula
\item Skolas urbanizācijas tipam -- joslu diagramma, kurā skolas stabiņš izcelts ar citu krāsu.
\end{enumerate}




\section{Pilnīgi izrēķinātie uzdevumi}

9--10 punktu vērtējumu ieguvušo uzdevumu īpatsvars no visiem attiecīgās skolas skolēnu vērtētajiem uzdevumiem. 

\begin{enumerate}
\item Skolas urbanizācijas tipam -- rangu tabula
\item Skolas urbanizācijas tiapm -- joslu diagramma, kurā skolas stabiņš izcelts ar citu krāsu.
\end{enumerate}


\section{Klašu grupas, uzdevumu tēmas}

Rezultāti pa skolēniem, klašu grupām, kā arī 4 apakšnozarēm (algebra, \v{g}eometrija, kombinatorika (ieskaitot algoritmiku), skaitļu teorija). Korelācija starp skolēna kopvērtējumu un vērtējumu par skolas uzdevumu. 

\begin{enumerate}
\item Katrā no klašu grupām, kurā piedalījās attiecīgās skolas skolēni, zīmējam histogrammu ar dalībnieku savāktajiem rezultātiem; konkrētās skolas skolēnus uzzīmējam pa virsu histogrammai kā krāsainus aplīšus. Blakus histogrammai - tabuliņas ar skolēnu skaitliskajiem rezultātiem (punkti pa uzdevumiem un summa). VIenas klašu grupas ietvaros aplīšus numurējam no labās uz kreiso pusi - t.i. labākos rezultātus norādām augšā. 
Pavisam var būt līdz 8 histogrammiņas + tabuliņas (katrai klašu grupai sava).
Katrai klašu grupai atzīmējam arī attiecīgās skolas+klašu grupas skolēnu aritmētisko vidējo - t.i. aplīšu masas centru.
\item Z-indekss (Z-score) katrā no nozarēm. Katrai no 4 nozarēm (teiksim, algebrai) un katram skolēnam izrēķinām viņa algebras uzdevuma vērtējuma Z-score attiecībā pret vidējo vērtējumu un vērtējuma standartnovirzi, ko šis uzdevums saņēma kopumā. Pēc tam visu konkrētās skolas skolēnu algebras Z-scores apkopojam un atrodam to aritmētisko vidējo. Iezīmējam šo rezultātu kā svītriņu kopīgā visu skolu histogrammā
\end{enumerate}




\end{document}
